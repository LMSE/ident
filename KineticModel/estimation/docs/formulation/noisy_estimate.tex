\documentclass[10pt]{report}

\usepackage{amsmath}
\usepackage{array}
\usepackage{tabu}
\usepackage{lmodern}

\begin{document}
	Optimization problem for estimation of kinetic parameters from noisy experimental data. The noisy data is generated through the use of additive noise in flux equations that use a known kinetic model to predict the responses of a metabolic network to different perturbations.\\
	\begin{equation}
	v^* = g(x,p) + {\cal{N}}(0,1)
	\end{equation}
	where ${\cal{N}}(0,1)$ is Gaussian noise with zero mean and a standard deviation of 1.
	\begin{equation}\label{eq:ode1}
	\frac{d}{dt}pep=v_1-v_2-v_4
	\end{equation}
	\begin{equation}\label{eq:ode2}
	\frac{d}{dt}fdp=v_2-v_3
	\end{equation}
	\begin{equation}\label{eq:ode3}
	\frac{d}{dt}E=v_{e,max}\left(\frac{1}{1+\left(\frac{fdp}{K_{e}^{fdp}}\right)^{n_e}}\right) - d E
	\end{equation}
	The kinetic expressions for fluxes $v_1$ through $v_4$ are given below. The consumption of acetate through $v_1$ and conversion of \textit{pep} through $v_2$ are expressed in Equations (\ref{eq:flux1}) and (\ref{eq:flux2}) respectively using Michaelis-Menten kinetics. The acetate flux through $v_1$ is also governed by the quantity of available enzyme E. 
	\begin{equation}\label{eq:flux1}
	v_1 = k_{1}^{cat}E\frac{acetate}{acetate+K_{1}^{acetate}}
	\end{equation}	
	\begin{equation}\label{eq:flux2}
	v_2 = V_{2}^{max}\frac{pep}{pep+K_{2}^{pep}}
	\end{equation}
	\begin{equation}\label{eq:flux3}
	v_3 = V_{3}^{max}\frac{\tilde{fdp}\left(1+\tilde{fdp}\right)^3}{\left(1+\tilde{fdp}\right)^4+L_3\left(1+\frac{pep}{K_{3}^{pep}}\right)^{-4}}
	\end{equation}
	The allosterically regulated flux $v_3$ for the consumption of \textit{fdp} is expressed in Equation (\ref{eq:flux3}) using the Monod-Wyman-Changeux (MWC) model for allosterically regulated enzymes, where $\tilde{fdp}$ refers to the ratio of \textit{fdp} with respect to its allosteric binding constant $K_{3}^{fdp}$. The added flux $v_4$ for the export of \textit{pep} is expressed as a linear equation dependent on $pep$ in Equation (\ref{eq:flux4}).
	\begin{equation}\label{eq:flux4}
	v_4 = k_{4}^{cat}.pep
	\end{equation}
	
	We denote the known noisy flux and concentration information with the superscript $^*$. Accordingly, fluxes $v^*$ and concentrations $x^*$ are the noisy information that will be used for estimation. Fluxes ($v$) and concentrations ($x$) that predicted by the model (without noise) are denoted without the superscript.
	
	The optimization formulation to estimate fluxes given below is based on the minimization of the tolerance for the L2-norm of the difference between the model predicted flux and the noisy flux. 
	\begin{center}
		\begin{subequations}\label{eq:noisyopt}
			\begin{align}
			\underset{x,p,e}{\mathrm{min}} & \text{      }e\\
			\mathrm{st} & \text{      }|v-v^*| \le \varepsilon\\
			& \text{      }|x-x^*| \le \varepsilon\\
			& \text{      }Sv = 0\\
			& \text{      }v = f(x,p) + e\\
			& \text{      }x_{min}\le x \le x_{max}\\
			& \text{      }p_{min} \le p \le p_{max}\\	
			& \text{      }e_{min} \le e \le e_{max}						
			\end{align}
		\end{subequations}		
	\end{center}

	In Equation (\ref{eq:noisyopt}) above, $\varepsilon$ is the absolute tolerance between the predicted and the measured quantities. The relationship between the predicted and the measured quantities are given by the nonlinear constraints in Equations (\ref{eq:noisyopt}b) and (\ref{eq:noisyopt}c). The stoichiometric constraints for steady state are given by Equation (\ref{eq:noisyopt}d), and Equation (\ref{eq:noisyopt}e) specifies the kinetic rate laws for the predicted fluxes as a function of the concentrations ($x$) and the parameters ($p$). Equations (\ref{eq:noisyopt}f) and (\ref{eq:noisyopt}g) are the bounds for the concentrations and parameters that are to be estimated by the optimization problem. The above optimization problem is to be solved for fixed values of $\varepsilon \in \{\text{.01, .05, .1}\}$.	
	
	\paragraph{Changes in problem formulation due to feasibility issues:}
	Since the above problem (Equation \ref{eq:noisyopt}) does not result in feasible global optimal solutions (using SCIP), the constraints are relaxed and the feasible problem is presented below.
	
	\begin{center}
		\begin{subequations}\label{eq:optmod}
			\begin{align}
			\underset{x,p_i,e_i}{\mathrm{min}} & \text{      }e_i\\
			\mathrm{st} & \text{      }|v_i-v^*| \le 0.2\\
			& \text{      }|x-x^*| \le 0.3\\
			& \text{      }|Sv| \le 1\text{x}10^{-8}\\
			& \text{      }v_i = f(x,p_i) + e\\
			& \text{      }v_{j\not= i} = f(x,p_{j\not=i})\\
			& \text{      }x_{min}\le x \le x_{max}\\
			& \text{      }p_{i,min} \le p_i \le p_{i,max}\\
			& \text{      }e_{i,min} \le e_i \le e_{i,max}					
			\end{align}
		\end{subequations}		
	\end{center}
	
	In Equation (\ref{eq:optmod}) above, the index $i$ represents the flux whose parameters are being optimized and index $j$ represents all other fluxes whose parameters are held constant. 
	
	The steady state constraint has been relaxed (from a equality constraint to an inequality constraint) along with permitting the concentrations and the estimated fluxes to deviate significantly from their experimentally observed values (~20 - 30\%). 
	
	The bounds for the flux noise were set at (0,20) in the above problem.
\end{document}

