\documentclass[10pt]{article}

\usepackage{amsmath}
\usepackage{amssymb}
\usepackage{array}
\usepackage{tabu}
\usepackage{lmodern}
\usepackage{graphicx}
\usepackage[space]{grffile}
\usepackage{subfigure}
\usepackage{longtable}
\usepackage{multirow}
\usepackage[margin=1.0in]{geometry}
\renewcommand{\baselinestretch}{2.0}

%\usepackage[font=small,labelfont=bf,labelsep=period]{caption}
\usepackage[style=authoryear,sorting=nyt,url=false,isbn=false,doi=false,firstinits=true,backend=biber]{biblatex}

\renewcommand{\baselinestretch}{2.0}
\renewcommand*\contentsname{Table of Contents}
%\captionsetup{font={stretch=2.0}}

\DeclareNameAlias{default}{last-first}

\DefineBibliographyStrings{english}{%
	andothers = {\addcomma\addspace\textsc{et\addabbrvspace al}\adddot},
	and = {\textsc{and}}
}
\renewcommand*{\labelnamepunct}{\space\space}

\renewbibmacro{in:}
{%
	\ifentrytype{article}{%
	}{%
		\printtext{\bibstring{in}\intitlepunct}%
	}%
}
\renewbibmacro*{volume+number}{%
	\printfield{volume}%
	\setunit*{\addcomma\space}%
	\printfield{number}%
	\setunit{\addcomma\space}}

\DeclareFieldFormat{pages}{#1}

\renewbibmacro*{publisher+location+date}{%
	\printlist{publisher}%
	\setunit*{\addcomma\space}%
	\printlist{location}%
	\setunit*{\addcomma\space}%
	\usebibmacro{date}%
	\newunit}

\renewcommand{\newunitpunct}{\addcomma\space}
\DeclareFieldFormat[article,inbook,incollection,inproceedings,patent,thesis,unpublished]{title}{#1} 
\DeclareFieldFormat{year}{#1} 

\addbibresource{refs/ident_refs.bib}

\begin{document}
	\section{Introduction:}
	\subsection{A method to establish posterior identifiability of metabolic network models:}
	This document details a method to establish the practical (posterior) identifiability of metabolic network models using the algebraic relationship between fluxes.
	Every flux, $v$, in a kinetic model of a metabolic network can be expressed as a nonlinear algebraic equation (Equation \ref{eq:flux_gen}). The fluxes are expressed as a function of the metabolite concentrations $x$ and the kinetic parameters $\theta$ in Equation (\ref{eq:flux_gen}).
	\begin{align}\label{eq:flux_gen}
	v = f(\mathbf{x},\mathbf{\theta})
	\end{align}
	Given the nonlinear nature of this model, the function $f$ in Equation (\ref{eq:flux_gen}) can expressed, without loss of generality as,
	\begin{align}\label{eq:nr_dr}
	v = \frac{N(\mathbf{x},\theta)}{D(\mathbf{x},\theta)}
	\end{align}
	where $N(\mathbf{x},\theta)$ is the numerator of $f$, and $D(\mathbf{x},\theta)$ is the denominator of $f$.
	
	If $\theta \in \mathbb{R}^p$, given a set of experimental measurements for the metabolite concentrations $\mathbf{x}$ and the reaction fluxes $\mathbf{v}$, theoretically, it is possible to choose $p$ sets of data from these measurements to solve for the $p$ parameters in $\theta$. However, if any these datasets do not satisfy the condition that $D(\mathbf{x},\theta) \neq 0$, then the number of experiments required to estimate the $p$ parameters in $\theta$ can be established to be greater than $p$. An example is shown below.
	
	This analysis can be performed for each flux in a metabolic network independent of all the other fluxes. This enables this method to be scalable to even genome-scale models.
\end{document}